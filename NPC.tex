

\documentclass[10pt,twoside,slovak,a4paper]{article}

\usepackage[slovak]{babel}
\usepackage[IL2]{fontenc} 
\usepackage[utf8]{inputenc}
\usepackage{graphicx}
\usepackage{url} 
\usepackage{hyperref} 

\usepackage{cite}

\pagestyle{headings}

\title{Vývoj, využitie a zdokonaľovanie Non-Player Characterov v hrách\thanks{Semestrálny projekt v predmete Metódy inžinierskej práce, ak. rok 2022/2023, vedenie: Ing. Igor Stupavský}}

%\title{Vývoj, využitie a zdokonaľovanie Non-Player Characterov v hrách}


\author{Peter Pipasík\\[2pt]
	{\small Slovenská technická univerzita v Bratislave}\\
	{\small Fakulta informatiky a informačných technológií}\\
	{\small \texttt{xpipasik@stuba.sk}}
	}

\date{\small 18. október 2022}

\begin{document}

\maketitle

\begin{abstract}
Je skoro nemožné predstaviť si hru bez iných postáv než je hráč či už sa jedná o postavu ktorá nám má v hre pomáhať, obchodovať s nami navigovať nás po hernom svete alebo bojovať proti nám. Každý z nás vníma tieto postavy ako samozrejmosť no nikto z nás sa nezamyslí nad tým ako by hry vyzerali bez postáv neovládaných hráčom (po anglicky non-player character, skrátene NPC). 
Herný svet by nemohol existovať bez týchto postáv a aj keby existoval nebol by tak zaujímavý, napínavý no občas aj predvídateľný. Moderné počítačové,konzolové a mobilné hry by bez NPCs boli ako filmy bez hercov. Nudné, nezáživné, neikonické v skratke bez NPCs by hry neboli tým čím sú. Skoro každá dobrá a hráčmi milovaná hra je tak úžasná prevažne vďaka jej deju a postavám ktoré tento dej prinášajú a spolu s hráčom vytvárajú.   

\end{abstract}

\section{Úvod}
V tomto článku sa budem snažiť obecne a následne technicky objasniť pojem NPC. Mojou prioritou bude vysvetlenie základov fungovania NPCs v hrách a ich využitie. Následne sa budem snažiť odvodiť dôležitosť týchto postáv v moderných hrách. Budem sa teda snažiť dokázať že moderné hry už nie sú schopné fungovať bez týchto postáv. Teoreticky odvodiť nemožnosť vytvorenia hry bez postáv typu NPC.
Obecné objasnenie pojmu NPC bude obsahovať vznik a využitie NPCs ale taktiež aj priblíženie čo je to NPC na základnej úrovni, teda jednoduché a jasné vysvetlenie základných pojmov spojených so slovným spojeným non-player character.
Hlbšie teda technické objasnenie pojmu NPC bude obsahovať postup a spôsob vytvorenia NPC. Teda bude obsahovať vysvetlenie umelej inteligencii pre NPC, pravidlá a ďalšie metódy a spôsoby potrebné na vytvorenie funkčného modelu NPC. 
K záveru chcem čitateľovi priblížiť pokrok za posledné roky. Presnejšie povedané históriu NPCs v hrách a pokrok vo vytváraní NPCs. Celkový pokrok obsahuje napríklad zlepšenie interakcie hráča s NPC postavou a rozvoj osobnostných čŕt pre NPCs. 


\section{Význam skratky NPC}  \label{Vyznam skratky}
Skratka NPC je odvodená z pojmov: Non-Player Character, Non-Player Class a taktiež Non-Playable Character. Jedná sa teda o nehráčsku alebo nehrateľnú postavu. Ako NPC sa považuje akákoľvek postava v hre ktorú neovláda hráč ale je ovládaná pomocou počítača teda ovláda ju samotná hra, herná mechanika a umelá inteligencia danej hry. V modernejších hrách má každá nehrateľná postava svoju vlastnú umelú inteligenciu. NPC má vopred určený súbor správania ktorý ovplyvňuje vyhodnocovanie daných situácií. NPC sa vyskytuje prevažne v hrách typu RPG (Role Playing Game) a MMORPG (Massively Multiplayer Online Role Playing Game).

\subsection{RPG} \label{RPG}

\subsection{MMORPG} \label{MMORPG}

\section{Vznik NPC}
\subsection{Prvý výskyt NPCs v hrách } \label{NPC 1 time}



\end{document}

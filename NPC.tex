

\documentclass[10pt,twoside,slovak,a4paper]{article}

\usepackage[slovak]{babel}
\usepackage[IL2]{fontenc} 
\usepackage[utf8]{inputenc}
\usepackage{graphicx}
\usepackage{url} 
\usepackage{hyperref} 

\usepackage{cite}

\pagestyle{headings}

\title{Vývoj, využitie a zdokonaľovanie Non-Player Characterov v hrách\thanks{Semestrálny projekt v predmete Metódy inžinierskej práce, ak. rok 2022/2023, vedenie: Ing. Igor Stupavský}}

%\title{Vývoj, využitie a zdokonaľovanie Non-Player Characterov v hrách}


\author{Peter Pipasík\\[2pt]
	{\small Slovenská technická univerzita v Bratislave}\\
	{\small Fakulta informatiky a informačných technológií}\\
	{\small \texttt{xpipasik@stuba.sk}}
	}

\date{\small 18. október 2022}

\begin{document}

\maketitle

% lepsie by bolo dat to opacne abstrakt a uvod... lebo uvod sa viac podoba abstraktu
% pretoze ten abstrakt pripomina skor uvod... nie je pisany tak ako by mal abstrakt byt pisany .....

\begin{abstract}
Je skoro nemožné predstaviť si hru bez iných postáv než je hráč či už sa jedná o postavu ktorá nám má v hre pomáhať, obchodovať s nami navigovať nás po hernom svete alebo bojovať proti nám. Každý z nás vníma tieto postavy ako samozrejmosť no nikto z nás sa nezamyslí nad tým ako by hry vyzerali bez postáv neovládaných hráčom (po anglicky non-player character, skrátene NPC). 
Herný svet by nemohol existovať bez týchto postáv a aj keby existoval nebol by tak zaujímavý, napínavý no občas aj predvídateľný. Moderné počítačové,konzolové a mobilné hry by bez NPCs boli ako filmy bez hercov. Nudné, nezáživné, neikonické v skratke bez NPCs by hry neboli tým čím sú. Skoro každá dobrá a hráčmi milovaná hra je tak úžasná prevažne vďaka jej deju a postavám ktoré tento dej prinášajú a spolu s hráčom vytvárajú.   

\end{abstract}

\section{Úvod}
V tomto článku sa budem snažiť obecne objasniť pojem NPC (časť~\ref{Vyznam skratky}). Mojou prioritou bude vysvetlenie základov fungovania NPCs v hrách a ich využitie. Následne sa budem snažiť odvodiť dôležitosť týchto postáv v moderných hrách. Budem sa teda snažiť dokázať že moderné hry už nie sú schopné fungovať bez týchto postáv. Teoreticky odvodiť nemožnosť vytvorenia hry bez postáv typu NPC.
Obecné objasnenie pojmu NPC bude obsahovať vznik (časť~\ref{Vznik}) a využitie NPCs ale taktiež aj priblíženie čo je to NPC na základnej úrovni, teda jednoduché a jasné vysvetlenie základných pojmov spojených so slovným spojeným non-player character (časť~\ref{Vyznam skratky}).
K záveru chcem čitateľovi priblížiť pokrok za posledné roky. Presnejšie povedané históriu NPCs v hrách a pokrok vo vytváraní NPCs. Celkový pokrok obsahuje napríklad zlepšenie interakcie hráča s NPC postavou a rozvoj osobnostných čŕt pre NPCs. 


\section{Význam skratky NPC}  \label{Vyznam skratky}
Skratka NPC je odvodená z pojmov: Non-Player Character, Non-Player Class a taktiež Non-Playable Character. Jedná sa teda o nehráčsku alebo nehrateľnú postavu. Ako NPC sa považuje akákoľvek postava v hre ktorú neovláda hráč ale je ovládaná pomocou počítača teda ovláda ju samotná hra, herná mechanika a umelá inteligencia danej hry. V modernejších hrách má každá nehrateľná postava svoju vlastnú umelú inteligenciu. NPC má vopred určený súbor správania ktorý ovplyvňuje vyhodnocovanie daných situácií. NPC sa vyskytuje prevažne v hrách typu RPG (Role Playing Game) a MMORPG (Massively Multiplayer Online Role Playing Game).

\subsection{RPG} \label{RPG}
% sem by sa dala dat ta skratka miesto hore...
\subsection{MMORPG} \label{MMORPG}
% sem by sa dala dat ta skratka miesto hore...



\section{Vznik NPC}    \label{Vznik}
\subsection{Prvý výskyt NPC} \label{NPC 1 time}
Odpovedať na otázku ktorá hra bola úplne prvou hrou obsahujúcou úplne prvé NPC je náročné. Niekto by mohol povedať že to bola hra Bertie the Brain z roku 1950 no však táto hra neobsahovala žiadne postavy. Dostatočne veľká skupina ľudí sa zhodla na tom že to bola hra Space Invaders z roku 1978. Táto hra je doposiaľ považovaná za úplne prvú hru ktorá splnila minimálne požiadavky na to aby mohla byť klasifikovaná ako hra obsahujúca NPC. Pointa hry bola jasná a jednoduchá a to zostreliť všetky mimozemské lode ktoré postupne klesali nižšie. Tieto mimozemské lode sú vnímané ako úplne prvé NPC v hrách, nakoľko hráč nemal možnosť ich ovládať a svojim spôsobom sa jednalo o postavy a nie len o umelú inteligenciu ako v prípade Bertie the Brain.  



\section{Typy NPCs}     \label{Typy}
V hrách sa často stretávame s mnohými NPCs ktoré sa prevažne delia na 3 typy a to priateľské NPCs (časť~\ref{Ally & non}), neutrálne NPCs (časť~\ref{Ally & non}) a nepriateľské NPCs (časť~\ref{enemy}) \cite{phdthesis}. Každá z týchto skupín má predom určenú úlohu a vopred určený význam v hre. Samozrejme záleží aj na príbehu danej hry. Tieto 3 typy sa dajú pokladať ako jedno z úplne základných rozdelení pre NPCs, pretože každý typ sa v závislosti na danej hre ešte ďalej rozvetvuje. Keďže každá hra je iná a v každej hre sa nachádzajú parametrálne odlišné NPCs som sa rozhodol opísať význam, správanie, dôležitosť a hlavnú funkciu najčastejšie sa vyskytujúcich podskupín. 

\subsection{Priateľské a neutrálne NPCs}  \label{Ally & non}  

\subsubsection{Civilisti} 
Civilisti v hrách majú jednu jedinú špecifickú úlohu a to vyplniť priestor aby mal hráč pocit že je vo svete ktorý skutočne žije. Ak je hráč v dostatočnej blízkosti NPC typu civilista zväčša má možnosť opýtať sa ho náhodnú otázku, pozdraviť ho alebo ho okradnúť v niektorých hrách ho dokonca aj zabiť. V prípade otázky a pozdravu od hráča mu NPC odpovie jednu z vopred naprogramovaných odpovedí. NPCs typu civilista sa prevažne vyskytujú v civilných oblastiach ktoré sa majú podobať mestám a dedinám v skutočnom živote.  


\subsubsection{Obchodníci}
NPCs typu obchodník slúžia hráčovi k nákupu zbraní, munície, máp a rôznych dekoračných premetov. Dalo by sa o nich povedať že plnia rolu predavača z reálneho sveta. Občas sú aj poskytovatelia služieb na vylepšenie hráčovej postavy. Tieto NPCs sú nesmierne dôležité pre každú hru. Niektoré hry obsahujú typ NPC ktorý plní rolu barmana, predajcu áut, predajcu pozemkov. Vo väčšine hier taktiež existuje typ obchodníka, ktorý poskytuje výmenu jeho inventáru za hernú menu. Vďaka hernej mene je hráč schopný nakúpiť si vyššie spomenuté predmety.  


\subsubsection{Poskytovatelia úloh, výziev a misií}
\subsubsection{Poskytovatelia základných informácií ohľadom histórie, tradícií a lokalít}


\subsubsection{Strážcovia a bojovníci} \label{services}
% co idu s tebou na misie alebo co brania urcitu oblast, ale aj taky co si ich vies kupit aby pre teba nico spravili

\subsection{Nepriateľské NPCs} \label{enemy}

\subsubsection{Hlavný protivník (Villain)} \label{BOSS}

\subsubsection{Boss} \label{BOSS}

\subsubsection{Pešiaci} \label{Pesiaci}

\subsubsection{Ostatný nepriatelia} \label{Pesiaci}

\bibliography{NPC_literatura}
\bibliographystyle{alpha}

\end{document}

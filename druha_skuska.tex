

\documentclass[10pt,twoside,slovak,a4paper]{article}

\usepackage[slovak]{babel}
\usepackage[IL2]{fontenc} 
\usepackage[utf8]{inputenc}

\title{Vývoj, využitie a zdokonaľovanie Non-Player Characterov v hrách}

\author{Peter Pipasík\\[2pt]
	{\small Slovenská technická univerzita v Bratislave}\\
	{\small Fakulta informatiky a informačných technológií}\\
	{\small \texttt{xpipasik@stuba.sk}}
	}

\date{\small 18. október 2022}

\begin{document}

\maketitle

\begin{abstract}
- sem treba dat abstrakt este
\end{abstract}

\section{úvod, čo je npc.....}
Skratka NPC je odvodená z pojmov: Non-Player Character, Non-Player Class a taktiež Non-Playable Character. Jedná sa teda o nehráčsku alebo nehrateľnú postavu. Ako NPC sa považuje akákoľvek postava v hre ktorú neovláda hráč ale je ovládaná pomocou počítača teda ovláda ju samotná hra, herná mechanika a umelá inteligencia danej hry. V modernejších hrách má každá nehrateľná postava svoju vlastnú umelú inteligenciu. NPC má vopred určený súbor správania ktorý ovplyvňuje vyhodnocovanie daných situácií. NPC sa vyskytuje prevažne v hrách typu RPG (Role Playing Game) a MMORPG (Massively Multiplayer Online Role Playing Game).

\section{Vznik NPC}


\end{document}
